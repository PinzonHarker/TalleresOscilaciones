\documentclass[a4paper]{article}
\usepackage{student}

% Metadata
\date{\today}
\setterm{Semestre 2022-1}

%-------------------------------%
% Other details
% TODO: Fill these
%-------------------------------%
\title{Taller: Semana 1}
\setmembername{Andrés Harker, Cristian Pérez}  % Fill group member names

%-------------------------------%
% Add / Delete commands and packages
% TODO: Add / Delete here as you need
%-------------------------------%
\usepackage{amsmath,amssymb,bm}
\usepackage[spanish]{babel}

% Custom your usual commands here. Renew these.
\newcommand{\KL}{\mathrm{KL}}
\newcommand{\R}{\mathbb{R}}
\newcommand{\E}{\mathbb{E}}
\newcommand{\T}{\top}
\newcommand{\expdist}[2]{%
        \normalfont{\textsc{Exp}}(#1, #2)%
    }
\newcommand{\expparam}{\bm \lambda}
\newcommand{\Expparam}{\bm \Lambda}
\newcommand{\natparam}{\bm \eta}
\newcommand{\Natparam}{\bm H}
\newcommand{\sufstat}{\bm u}

% Main document
\begin{document}
    % Add header
    \header{}

    % Use `answer` environment to add solutions
    % \begin{answer}[Question 1.1] for example starts an environment formatted
    % for Question 1.1
    \begin{answer}[Problema 1.]
    \begin{enumerate}
        \item [a)] sea $x(t) = ACos(\omega t + \phi)$ entonces
        \begin{equation*}
        \begin{split}
             m \ddot{x} &= -mA\omega^2 Cos(\omega t + \phi)\\
            &=  -m\frac{k}{m}A Cos(\omega t + \phi)\\
            &=  -k A Cos(\omega t + \phi)\\
             &=  -kx\\
        \end{split}
        \end{equation*}
            donde $\omega = \sqrt{k/m}$
              \item [b)] sea $x(t) = BSin(\omega t) + CCos(\omega t)$ entonces
        \begin{equation*}
        \begin{split}
             m \ddot{x} &= -mB\omega^2Sin(\omega t) - mC\omega^2Cos(\omega t)\\
            &= -mB\frac{k}{m}Sin(\omega t) - mC\frac{k}{m}Cos(\omega t)\\
            &= -BkSin(\omega t) - CmCos(\omega t)\\
             &=  -kx\\
        \end{split}
        \end{equation*}
            donde $\omega = \sqrt{k/m}$
            
            
    \end{enumerate}
    \end{answer}
    
    \begin{answer}[Problema 2.]
        Haciendo la expansión del coseno de la suma, tenemos lo siguiente:
        \begin{equation}
            \begin{split}
                A Cos(\omega t + \phi) = A(Cos(\omega t)Cos(\phi) - Sen(\omega t)Sen(\phi))\\
                = [ACos(\phi)]Cos(\omega t) + [-ASen(\omega t]Sen(\omega t)
            \end{split}
        \end{equation}
        Por tanto, nuestro $A$ y $B$ en la segunda ecuación son determinados por una constante $A=A'$ en la primera, luego la relación es:
        \begin{equation}
            A = [A'Cos(\phi)]\\
            B = [-A'Sen(\omega t]
        \end{equation}
    
    \end{answer}
    
     \begin{answer}[Problema 3.]
     
     \begin{enumerate}
\item[a)] Si $x(t) = ACos(\omega t. + \phi)$ entonces:
$$x(0) = ACos(\phi) = x_0$$ 
$$v(0) = -A\omega Sin(\phi) = v_0$$
Luego

$$\frac{v_0}{x_0} = -\omega Tan(\phi)\hspace{0.35cm}\Rightarrow$$
$$-\frac{v_0}{\omega x_0} =  Tan(\phi)\hspace{0.35cm}\Rightarrow$$
$$ArcTan\left(-\frac{v_0}{\omega x_0}\right) =  \phi\hspace{0.35cm}\Rightarrow$$

Y ademas 
 
$$x_0^2 + \frac{v_0^2}{\omega^2} = A^2Cos^2(\phi) + A^2Sin^2(\phi)\hspace{0.35cm}\Rightarrow $$ 
$$x_0^2 + \frac{v_0^2}{\omega^2} = A^2\hspace{0.35cm}\Rightarrow $$ 
$$A = \sqrt{x_0^2 + \frac{v_0^2}{\omega^2}}$$   

\item[b)] Si 
$x(t) = BSin(\omega t) + CCos(\omega t) $ entonces:
$$x(0) = C = x_0$$ 
$$v(0) = B\omega = v_0$$
Luego

$$C = x_0$$
$$B = \frac{v_0}{\omega}$$

\end{enumerate}
     \end{answer}
         % Problema 4.
    \begin{answer}[Problema 4.]
        Dado que la fuerza $F=mg$ en algún momento es igual a $F_r=-kx$, existe un punto de equilibrio de tal manera que hay una fuerza equivalente $F_e = F_r - F_g$. Luego la fuerza resultante es equivalente a un sistema de resorte vertical (asumiendo rozamientos y fuerzas externas despreciables).
    \end{answer}
     
    \begin{answer}[Problema 5.]
    
        Del esquema de fuerzas de la figura 1 obtenemos que:
        
        
        \begin{equation*}
            ml\ddot{\theta} = -mgSen(\theta)
        \end{equation*}
        Luego:
\begin{equation*}
     \ddot{\theta} + \frac{g}{l}Sen(\theta) = 0
\end{equation*}
y tomando $\theta$ pequeño  entonces:

    \begin{equation*}
     \ddot{\theta} + \frac{g}{l}\theta = 0
\end{equation*}
Cuya solucion es de la forma:
    \begin{equation*}
        \theta(t) = ACos(\omega t + \phi)
    \end{equation*}
    
    Donde:
    \begin{equation*}
        \dot{\theta} = -\omega ASin(\omega t + \phi)
    \end{equation*}
    Por lo cual su energia cinetica queda como 
    $$K = \frac{1}{2}ml^2\omega^2 A^2 Sin^2(\omega t+\phi )  $$
    
    Y ademas como la energia potencial graviatacional es de la forma $U = mgy$
    entonces la energia potencial para el caso del pendulo quedara como:
    
    \begin{equation*}
        U = mglCos(\theta)
    \end{equation*}
    \end{answer}
    
        % Problema 6.
    \begin{answer}[Problema 6.]
        Teniendo en cuenta el estado en equilibrio $m = 0.1 kg$ y $x = 0.1m$ con la siguiente ecuación:
            \begin{equation}
                mg = F_g = F_r = kx =\\
                (0.1 kg)(9.8 m/s^2) = k (0.1m)\\
                k = 9.8 N/m
            \end{equation}
        \begin{enumerate}
            \item [a).] Dado que $\omega^2 = k/m$, encontramos que la ecuación diferencial del movimiento resulta:
            $$d^2x/dt^2 + 98 x = 0$$
            \item [b).] Dado el estado inicial, por la ecuación armónica:
            $$x(t) = ACos(\omega t + \phi)$$
            Tenemos $A = 6cm = 0.06 m$, $\phi = 0$ dado que parte de reposo y $v = 0$. Luego:
            $$ x(t) = 0.06 Cos(7 \sqrt{2} t)$$
            
        \end{enumerate}
    \end{answer}
    
        % Problema 7.
    \begin{answer}[Problema 7.]
    
        Dado que el periodo de un pendulo es: 
        \begin{equation*}
            T = 2\pi \sqrt{\frac{l}{g}}
        \end{equation*}
        Entonces el tiempo requerido para que el pendulo de un metro complete una oscilacion es:
        $$12 T_1 = 24\pi\sqrt{1/9.82} = 24.1seg$$
        por tanto el tiempo que tarda el pendulo de longuitud desconocida en completar 11 oscilaciones es $24.1seg$ luego:
        
       $$11 T_2 = 24.1 = 22\pi \sqrt{\frac{l_2}{9.82}}$$
       
       entonces: 
       $$l_2 = \left(\frac{24.1}{22\pi}\right)^2(9.82) = 1.2 m$$
    \end{answer}
    
    % Problema 8.
    \begin{answer}[8.]
    Teniendo en cuenta la energía potencial de la gravedad $U = mgh$ y que la energía del resorte en el punto más alto es $U = 1/2 kx^2$. Igualamos la energía:
    $$\omega^2 = k/m = gh/x^2 = (9.8)(0.49)/(0.245)^2$$
    $$ \omega^2 = 80 $$
    Dado que $\omega = 2\pi f$ tenemos:
    $$ f = 8.94/2\pi = 1.423 Hz$$
    o $T =  0.703 s$
    
    \end{answer}
    
    % Problema 9.
    \begin{answer}[Problema 9.]
    De la figura 2 vemos que si la masa esta en equilibrio:
    $$ky = mg\hspace{0.35cm}\Rightarrow$$
    $$k = \frac{mg}{y}\hspace{0.35cm}\Rightarrow$$
    $$k = \frac{0.1(9.82) }{9.82} = 0.1\hspace{0.35cm}\Rightarrow$$
    Y por tanto si adiciona una masa de $0.2kg$
    entonces el resorte se elongara:
    $$y = \frac{0.3(9,82)}{0.1} = 29.46m$$
    \end{answer}
    
    % Problema 10.
    \begin{answer}[10).]
    
    \end{answer}
\end{document}
